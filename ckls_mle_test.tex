% Options for packages loaded elsewhere
\PassOptionsToPackage{unicode}{hyperref}
\PassOptionsToPackage{hyphens}{url}
%
\documentclass[
]{article}
\usepackage{amsmath,amssymb}
\usepackage{lmodern}
\usepackage{ifxetex,ifluatex}
\ifnum 0\ifxetex 1\fi\ifluatex 1\fi=0 % if pdftex
  \usepackage[T1]{fontenc}
  \usepackage[utf8]{inputenc}
  \usepackage{textcomp} % provide euro and other symbols
\else % if luatex or xetex
  \usepackage{unicode-math}
  \defaultfontfeatures{Scale=MatchLowercase}
  \defaultfontfeatures[\rmfamily]{Ligatures=TeX,Scale=1}
\fi
% Use upquote if available, for straight quotes in verbatim environments
\IfFileExists{upquote.sty}{\usepackage{upquote}}{}
\IfFileExists{microtype.sty}{% use microtype if available
  \usepackage[]{microtype}
  \UseMicrotypeSet[protrusion]{basicmath} % disable protrusion for tt fonts
}{}
\makeatletter
\@ifundefined{KOMAClassName}{% if non-KOMA class
  \IfFileExists{parskip.sty}{%
    \usepackage{parskip}
  }{% else
    \setlength{\parindent}{0pt}
    \setlength{\parskip}{6pt plus 2pt minus 1pt}}
}{% if KOMA class
  \KOMAoptions{parskip=half}}
\makeatother
\usepackage{xcolor}
\IfFileExists{xurl.sty}{\usepackage{xurl}}{} % add URL line breaks if available
\IfFileExists{bookmark.sty}{\usepackage{bookmark}}{\usepackage{hyperref}}
\hypersetup{
  pdftitle={CKLS 1D - Sim.DiffProc},
  pdfauthor={R. Dance},
  hidelinks,
  pdfcreator={LaTeX via pandoc}}
\urlstyle{same} % disable monospaced font for URLs
\usepackage[margin=1in]{geometry}
\usepackage{color}
\usepackage{fancyvrb}
\newcommand{\VerbBar}{|}
\newcommand{\VERB}{\Verb[commandchars=\\\{\}]}
\DefineVerbatimEnvironment{Highlighting}{Verbatim}{commandchars=\\\{\}}
% Add ',fontsize=\small' for more characters per line
\usepackage{framed}
\definecolor{shadecolor}{RGB}{248,248,248}
\newenvironment{Shaded}{\begin{snugshade}}{\end{snugshade}}
\newcommand{\AlertTok}[1]{\textcolor[rgb]{0.94,0.16,0.16}{#1}}
\newcommand{\AnnotationTok}[1]{\textcolor[rgb]{0.56,0.35,0.01}{\textbf{\textit{#1}}}}
\newcommand{\AttributeTok}[1]{\textcolor[rgb]{0.77,0.63,0.00}{#1}}
\newcommand{\BaseNTok}[1]{\textcolor[rgb]{0.00,0.00,0.81}{#1}}
\newcommand{\BuiltInTok}[1]{#1}
\newcommand{\CharTok}[1]{\textcolor[rgb]{0.31,0.60,0.02}{#1}}
\newcommand{\CommentTok}[1]{\textcolor[rgb]{0.56,0.35,0.01}{\textit{#1}}}
\newcommand{\CommentVarTok}[1]{\textcolor[rgb]{0.56,0.35,0.01}{\textbf{\textit{#1}}}}
\newcommand{\ConstantTok}[1]{\textcolor[rgb]{0.00,0.00,0.00}{#1}}
\newcommand{\ControlFlowTok}[1]{\textcolor[rgb]{0.13,0.29,0.53}{\textbf{#1}}}
\newcommand{\DataTypeTok}[1]{\textcolor[rgb]{0.13,0.29,0.53}{#1}}
\newcommand{\DecValTok}[1]{\textcolor[rgb]{0.00,0.00,0.81}{#1}}
\newcommand{\DocumentationTok}[1]{\textcolor[rgb]{0.56,0.35,0.01}{\textbf{\textit{#1}}}}
\newcommand{\ErrorTok}[1]{\textcolor[rgb]{0.64,0.00,0.00}{\textbf{#1}}}
\newcommand{\ExtensionTok}[1]{#1}
\newcommand{\FloatTok}[1]{\textcolor[rgb]{0.00,0.00,0.81}{#1}}
\newcommand{\FunctionTok}[1]{\textcolor[rgb]{0.00,0.00,0.00}{#1}}
\newcommand{\ImportTok}[1]{#1}
\newcommand{\InformationTok}[1]{\textcolor[rgb]{0.56,0.35,0.01}{\textbf{\textit{#1}}}}
\newcommand{\KeywordTok}[1]{\textcolor[rgb]{0.13,0.29,0.53}{\textbf{#1}}}
\newcommand{\NormalTok}[1]{#1}
\newcommand{\OperatorTok}[1]{\textcolor[rgb]{0.81,0.36,0.00}{\textbf{#1}}}
\newcommand{\OtherTok}[1]{\textcolor[rgb]{0.56,0.35,0.01}{#1}}
\newcommand{\PreprocessorTok}[1]{\textcolor[rgb]{0.56,0.35,0.01}{\textit{#1}}}
\newcommand{\RegionMarkerTok}[1]{#1}
\newcommand{\SpecialCharTok}[1]{\textcolor[rgb]{0.00,0.00,0.00}{#1}}
\newcommand{\SpecialStringTok}[1]{\textcolor[rgb]{0.31,0.60,0.02}{#1}}
\newcommand{\StringTok}[1]{\textcolor[rgb]{0.31,0.60,0.02}{#1}}
\newcommand{\VariableTok}[1]{\textcolor[rgb]{0.00,0.00,0.00}{#1}}
\newcommand{\VerbatimStringTok}[1]{\textcolor[rgb]{0.31,0.60,0.02}{#1}}
\newcommand{\WarningTok}[1]{\textcolor[rgb]{0.56,0.35,0.01}{\textbf{\textit{#1}}}}
\usepackage{graphicx}
\makeatletter
\def\maxwidth{\ifdim\Gin@nat@width>\linewidth\linewidth\else\Gin@nat@width\fi}
\def\maxheight{\ifdim\Gin@nat@height>\textheight\textheight\else\Gin@nat@height\fi}
\makeatother
% Scale images if necessary, so that they will not overflow the page
% margins by default, and it is still possible to overwrite the defaults
% using explicit options in \includegraphics[width, height, ...]{}
\setkeys{Gin}{width=\maxwidth,height=\maxheight,keepaspectratio}
% Set default figure placement to htbp
\makeatletter
\def\fps@figure{htbp}
\makeatother
\setlength{\emergencystretch}{3em} % prevent overfull lines
\providecommand{\tightlist}{%
  \setlength{\itemsep}{0pt}\setlength{\parskip}{0pt}}
\setcounter{secnumdepth}{-\maxdimen} % remove section numbering
\ifluatex
  \usepackage{selnolig}  % disable illegal ligatures
\fi

\title{CKLS 1D - Sim.DiffProc}
\author{R. Dance}
\date{}

\begin{document}
\maketitle

{
\setcounter{tocdepth}{2}
\tableofcontents
}
This is an \href{http://rmarkdown.rstudio.com}{R Markdown} Notebook.
When you execute code within the notebook, the results appear beneath
the code.

Try executing this chunk by clicking the \emph{Run} button within the
chunk or by placing your cursor inside it and pressing
\emph{Cmd+Shift+Enter}.

Add a new chunk by clicking the \emph{Insert Chunk} button on the
toolbar or by pressing \emph{Cmd+Option+I}.

When you save the notebook, an HTML file containing the code and output
will be saved alongside it (click the \emph{Preview} button or press
\emph{Cmd+Shift+K} to preview the HTML file).

The preview shows you a rendered HTML copy of the contents of the
editor. Consequently, unlike \emph{Knit}, \emph{Preview} does not run
any R code chunks. Instead, the output of the chunk when it was last run
in the editor is displayed.

This chunk is the CKLS SDE with some initial conditions in it!
\href{https://cran.r-project.org/web/packages/Sim.DiffProc/vignettes/fitsde.html}{Link
to Source}

SET UP AS AN OU for a match to my Excel doc\ldots{} the original
parameters from the website are commented out next to the variables for
the OU.

CKLS:

\(dX_t = (\theta_1 +\theta_2X_t)dt + \theta_3X_t^{\theta_4}dW_t\)

where \(X_0=0.2\), and \(\theta_1=a0 = 0.01\), \(\theta_2 = b0 = -0.1\),
\(\theta_3 = s0 = 0.1\) and \(\theta_4 = g0 = 0\).

\begin{Shaded}
\begin{Highlighting}[]
\FunctionTok{library}\NormalTok{(Sim.DiffProc)}
\end{Highlighting}
\end{Shaded}

\begin{verbatim}
## Package 'Sim.DiffProc', version 4.8
## browseVignettes('Sim.DiffProc') for more informations.
\end{verbatim}

\begin{Shaded}
\begin{Highlighting}[]
\FunctionTok{set.seed}\NormalTok{(}\DecValTok{12345}\NormalTok{, }\AttributeTok{kind=}\StringTok{"L\textquotesingle{}Ecuyer{-}CMRG"}\NormalTok{)}
\NormalTok{a0 }\OtherTok{\textless{}{-}} \FloatTok{0.01} \CommentTok{\#1}
\NormalTok{b0 }\OtherTok{\textless{}{-}} \SpecialCharTok{{-}}\FloatTok{0.1} \CommentTok{\#2}
\NormalTok{s0 }\OtherTok{\textless{}{-}} \FloatTok{0.1} \CommentTok{\#0.5}
\NormalTok{g0 }\OtherTok{\textless{}{-}} \DecValTok{0} \CommentTok{\#0.3}
\CommentTok{\# I think this is simulated data}
\NormalTok{f }\OtherTok{\textless{}{-}} \FunctionTok{expression}\NormalTok{( (a0 }\SpecialCharTok{+}\NormalTok{ b0}\SpecialCharTok{*}\NormalTok{x) )}
\NormalTok{g }\OtherTok{\textless{}{-}} \FunctionTok{expression}\NormalTok{(s0 }\SpecialCharTok{*}\NormalTok{ x}\SpecialCharTok{\^{}}\NormalTok{g0)}
\NormalTok{sim }\OtherTok{\textless{}{-}} \FunctionTok{snssde1d}\NormalTok{(}\AttributeTok{drift=}\NormalTok{f, }\AttributeTok{diffusion=}\NormalTok{g, }\AttributeTok{x0=}\FloatTok{0.2}\NormalTok{, }\AttributeTok{N=}\DecValTok{10}\SpecialCharTok{\^{}}\DecValTok{3}\NormalTok{, }\AttributeTok{Dt=}\DecValTok{10}\SpecialCharTok{\^{}{-}}\DecValTok{3}\NormalTok{)}
\NormalTok{mydata }\OtherTok{\textless{}{-}}\NormalTok{ sim}\SpecialCharTok{$}\NormalTok{X}
\CommentTok{\# This is the model}
\NormalTok{fx }\OtherTok{\textless{}{-}} \FunctionTok{expression}\NormalTok{( theta[}\DecValTok{1}\NormalTok{]}\SpecialCharTok{+}\NormalTok{theta[}\DecValTok{2}\NormalTok{]}\SpecialCharTok{*}\NormalTok{x ) }\DocumentationTok{\#\# drift coefficient of model}
\NormalTok{gx }\OtherTok{\textless{}{-}} \FunctionTok{expression}\NormalTok{( theta[}\DecValTok{3}\NormalTok{]}\SpecialCharTok{*}\NormalTok{x}\SpecialCharTok{\^{}}\NormalTok{theta[}\DecValTok{4}\NormalTok{] ) }\DocumentationTok{\#\# diffusion coefficient of model}

\NormalTok{fitmod }\OtherTok{\textless{}{-}} \FunctionTok{fitsde}\NormalTok{(}\AttributeTok{data =}\NormalTok{ mydata, }\AttributeTok{drift =}\NormalTok{ fx, }\AttributeTok{diffusion =}\NormalTok{ gx, }\AttributeTok{start =} \FunctionTok{list}\NormalTok{(}\AttributeTok{theta1=}\FloatTok{0.05}\NormalTok{, }\AttributeTok{theta2=}\FloatTok{0.05}\NormalTok{, }\AttributeTok{theta3=}\FloatTok{0.05}\NormalTok{,}\AttributeTok{theta4=}\DecValTok{0}\NormalTok{),}\AttributeTok{pmle=}\StringTok{"euler"}\NormalTok{)}
\end{Highlighting}
\end{Shaded}

\begin{verbatim}
## Warning in dnorm(x, mean = x0 + A(t0, x0, theta) * dt, sd = sqrt(dt) * S(t0, :
## NaNs produced

## Warning in dnorm(x, mean = x0 + A(t0, x0, theta) * dt, sd = sqrt(dt) * S(t0, :
## NaNs produced

## Warning in dnorm(x, mean = x0 + A(t0, x0, theta) * dt, sd = sqrt(dt) * S(t0, :
## NaNs produced

## Warning in dnorm(x, mean = x0 + A(t0, x0, theta) * dt, sd = sqrt(dt) * S(t0, :
## NaNs produced

## Warning in dnorm(x, mean = x0 + A(t0, x0, theta) * dt, sd = sqrt(dt) * S(t0, :
## NaNs produced

## Warning in dnorm(x, mean = x0 + A(t0, x0, theta) * dt, sd = sqrt(dt) * S(t0, :
## NaNs produced

## Warning in dnorm(x, mean = x0 + A(t0, x0, theta) * dt, sd = sqrt(dt) * S(t0, :
## NaNs produced

## Warning in dnorm(x, mean = x0 + A(t0, x0, theta) * dt, sd = sqrt(dt) * S(t0, :
## NaNs produced

## Warning in dnorm(x, mean = x0 + A(t0, x0, theta) * dt, sd = sqrt(dt) * S(t0, :
## NaNs produced

## Warning in dnorm(x, mean = x0 + A(t0, x0, theta) * dt, sd = sqrt(dt) * S(t0, :
## NaNs produced

## Warning in dnorm(x, mean = x0 + A(t0, x0, theta) * dt, sd = sqrt(dt) * S(t0, :
## NaNs produced

## Warning in dnorm(x, mean = x0 + A(t0, x0, theta) * dt, sd = sqrt(dt) * S(t0, :
## NaNs produced

## Warning in dnorm(x, mean = x0 + A(t0, x0, theta) * dt, sd = sqrt(dt) * S(t0, :
## NaNs produced

## Warning in dnorm(x, mean = x0 + A(t0, x0, theta) * dt, sd = sqrt(dt) * S(t0, :
## NaNs produced

## Warning in dnorm(x, mean = x0 + A(t0, x0, theta) * dt, sd = sqrt(dt) * S(t0, :
## NaNs produced

## Warning in dnorm(x, mean = x0 + A(t0, x0, theta) * dt, sd = sqrt(dt) * S(t0, :
## NaNs produced

## Warning in dnorm(x, mean = x0 + A(t0, x0, theta) * dt, sd = sqrt(dt) * S(t0, :
## NaNs produced

## Warning in dnorm(x, mean = x0 + A(t0, x0, theta) * dt, sd = sqrt(dt) * S(t0, :
## NaNs produced
\end{verbatim}

\begin{Shaded}
\begin{Highlighting}[]
\FunctionTok{plot}\NormalTok{(sim)}
\end{Highlighting}
\end{Shaded}

\includegraphics{ckls_mle_test_files/figure-latex/unnamed-chunk-1-1.pdf}

\begin{Shaded}
\begin{Highlighting}[]
\FunctionTok{coef}\NormalTok{(fitmod) }
\end{Highlighting}
\end{Shaded}

\begin{verbatim}
##     theta1     theta2     theta3     theta4 
##  1.7854664 -9.9576713  0.1125686  0.1005015
\end{verbatim}

\begin{Shaded}
\begin{Highlighting}[]
\FunctionTok{summary}\NormalTok{(fitmod) }
\end{Highlighting}
\end{Shaded}

\begin{verbatim}
## Pseudo maximum likelihood estimation
## 
## Method:  Euler
## Call:
## fitsde(data = mydata, drift = fx, diffusion = gx, start = list(theta1 = 0.05, 
##     theta2 = 0.05, theta3 = 0.05, theta4 = 0), pmle = "euler")
## 
## Coefficients:
##          Estimate Std. Error
## theta1  1.7854664 0.79108015
## theta2 -9.9576713 4.36268019
## theta3  0.1125686 0.03543815
## theta4  0.1005015 0.18276921
## 
## -2 log L: -8783.792
\end{verbatim}

So this gives a log likelihood (divide this by -2), which is a half
decent match to the Excel, HOWEVER, the parameters are wildly off!?!
Look at the theta1 and theta2, they are way out, but the standard errors
are also huge??

vcov produces the variance-covariance matrix for the parameters\ldots{}

\begin{Shaded}
\begin{Highlighting}[]
\FunctionTok{vcov}\NormalTok{(fitmod)}
\end{Highlighting}
\end{Shaded}

\begin{verbatim}
##               theta1       theta2        theta3       theta4
## theta1  0.6258078047 -3.426414532  0.0008666953  0.004476970
## theta2 -3.4264145322 19.032978458 -0.0047887562 -0.024731572
## theta3  0.0008666953 -0.004788756  0.0012558622  0.006460655
## theta4  0.0044769700 -0.024731572  0.0064606545  0.033404584
\end{verbatim}

Log Likelihood\ldots.

\begin{Shaded}
\begin{Highlighting}[]
\FunctionTok{logLik}\NormalTok{(fitmod)}
\end{Highlighting}
\end{Shaded}

\begin{verbatim}
## [1] 4391.896
\end{verbatim}

Not too shabby as I got 4300 in Excel, but why are the parameters so
bad?!

I dont know what AIC (Akaike information criterion) or BIC (Bayesian
information criterion) are in practice\ldots{}

\begin{Shaded}
\begin{Highlighting}[]
\FunctionTok{AIC}\NormalTok{(fitmod)}
\end{Highlighting}
\end{Shaded}

\begin{verbatim}
## [1] -8775.792
\end{verbatim}

\begin{Shaded}
\begin{Highlighting}[]
\FunctionTok{BIC}\NormalTok{(fitmod)}
\end{Highlighting}
\end{Shaded}

\begin{verbatim}
## [1] -8769.975
\end{verbatim}

Confidence Intervals\ldots{} expecting this to be hilariously bad
considering theta1 and theta2 above\ldots{}

\begin{Shaded}
\begin{Highlighting}[]
\FunctionTok{confint}\NormalTok{(fitmod,}\AttributeTok{level=}\FloatTok{0.95}\NormalTok{)}
\end{Highlighting}
\end{Shaded}

\begin{verbatim}
##               2.5 %     97.5 %
## theta1   0.23497776  3.3359550
## theta2 -18.50836738 -1.4069753
## theta3   0.04311106  0.1820260
## theta4  -0.25771957  0.4587226
\end{verbatim}

Again, the theta3, which is my sigma, is fine. The theta2 isn't so bad
here - its comparing -0.1 to 0.7 which although bad isn't as bad as
theta1, which is wildly out. ANY IDEAS?!

\_

\end{document}
